\documentclass[12pt]{article}
\usepackage[margin=1in]{geometry}
\usepackage{graphicx}
\usepackage{listings}
\usepackage{xcolor}

\lstset{
    language=Matlab,
    basicstyle=\ttfamily\small,
    keywordstyle=\color{blue},
    commentstyle=\color{gray},
    stringstyle=\color{red},
    numbers=left,
    numberstyle=\tiny,
    frame=single,
    breaklines=true
}

\title{Assignment: Entropy Weight Method (EWM) Analysis}
\author{Your Name \\ Course Title}
\date{\today}

\begin{document}
\maketitle

\section*{Question 1}
A regional government is evaluating the efficiency of 10 local hospitals (DMUs) in managing their operations. Each hospital uses three key resources (inputs):
\begin{itemize}
  \item Input 1: Number of doctors
  \item Input 2: Number of nurses
  \item Input 3: Number of available hospital beds
\end{itemize}
The data for the 10 hospitals is shown below:
\begin{center}
\begin{tabular}{c|ccc}
DMU & Doctors & Nurses & Beds \\hline
DMU1  & 12 & 25 & 40 \\
DMU2  & 18 & 30 & 35 \\
DMU3  & 10 & 20 & 45 \\
DMU4  & 15 & 28 & 38 \\
DMU5  & 20 & 32 & 50 \\
DMU6  & 14 & 22 & 42 \\
DMU7  & 16 & 27 & 36 \\
DMU8  & 11 & 24 & 48 \\
DMU9  & 19 & 29 & 41 \\
DMU10 & 13 & 21 & 39 \\
\end{tabular}
\end{center}
Use the Entropy Weight Method (EWM) to:
\begin{enumerate}
  \item Normalize the data for all three inputs.
  \item Calculate the entropy value for each input.
  \item Compute the final weights for the three inputs.
\end{enumerate}

\subsection*{MATLAB Code}
\begin{lstlisting}
data = [
    12 25 40;
    18 30 35;
    10 20 45;
    15 28 38;
    20 32 50;
    14 22 42;
    16 27 36;
    11 24 48;
    19 29 41;
    13 21 39
];

X = data;
X_norm = X ./ sum(X);

[m, n] = size(X_norm);
k = 1/log(m);
entropy = -k * sum(X_norm .* log(X_norm + eps));

d = 1 - entropy;
w = d / sum(d);

disp('Entropy weights for inputs [Doctors, Nurses, Beds]:');
disp(w);
\end{lstlisting}

\subsection*{Output Screenshot}
\begin{figure}[h]
  \centering
  \includegraphics[width=0.7\textwidth]{ss1.png}
  \caption{MATLAB command window output for Q1}
\end{figure}

\newpage
\section*{Question 2}
A financial institution is evaluating the efficiency of 10 bank branches (DMUs). Each branch uses four resources:
\begin{itemize}
  \item Input 1: Number of employees
  \item Input 2: Operational cost (in million $)
  \item Input 3: Number of service counters
  \item Input 4: Number of ATMs
\end{itemize}
The data for the 10 branches is:
\begin{center}
\begin{tabular}{c|cccc}
DMU & Employees & Cost & Counters & ATMs \\hline
DMU1  & 45 & 1.80 & 12 & 4 \\
DMU2  & 52 & 2.00 & 15 & 6 \\
DMU3  & 40 & 1.50 & 11 & 3 \\
DMU4  & 55 & 2.20 & 14 & 5 \\
DMU5  & 48 & 1.90 & 13 & 4 \\
DMU6  & 60 & 2.50 & 16 & 7 \\
DMU7  & 43 & 1.70 & 12 & 3 \\
DMU8  & 50 & 2.10 & 15 & 6 \\
DMU9  & 47 & 1.80 & 13 & 5 \\
DMU10 & 53 & 2.30 & 14 & 6 \\
\end{tabular}
\end{center}
Use EWM to determine the weights of the four inputs.

\subsection*{MATLAB Code}
\begin{lstlisting}
data = [
    45 1.80 12 4;
    52 2.00 15 6;
    40 1.50 11 3;
    55 2.20 14 5;
    48 1.90 13 4;
    60 2.50 16 7;
    43 1.70 12 3;
    50 2.10 15 6;
    47 1.80 13 5;
    53 2.30 14 6
];

X_norm = data ./ sum(data);

[m, n] = size(X_norm);
k = 1/log(m);
entropy = -k * sum(X_norm .* log(X_norm + eps));

d = 1 - entropy;
w = d / sum(d);

disp('Entropy weights for inputs [Employees, Cost, Counters, ATMs]:');
disp(w);
\end{lstlisting}

\subsection*{Output Screenshot}
\begin{figure}[h]
  \centering
  \includegraphics[width=0.7\textwidth]{ss2.png}
  \caption{MATLAB command window output for Q2}
\end{figure}

\newpage
\section*{Question 3}
A university is analyzing the resource allocation of 10 academic departments (DMUs). Each department uses five inputs:
\begin{itemize}
  \item Input 1: Number of faculty members
  \item Input 2: Annual research funding (in $100,000)
  \item Input 3: Number of classrooms
  \item Input 4: Number of graduate students
  \item Input 5: Number of publications per year
\end{itemize}
The collected data is:
\begin{center}
\begin{tabular}{c|ccccc}
DMU & Faculty & Funding & Classrooms & Grad Students & Publications \\hline
DMU1  & 30 & 12 & 8  & 50 & 20 \\
DMU2  & 35 & 15 & 10 & 65 & 28 \\
DMU3  & 28 & 11 & 7  & 40 & 15 \\
DMU4  & 32 & 14 & 9  & 55 & 22 \\
DMU5  & 40 & 18 & 11 & 80 & 35 \\
DMU6  & 31 & 13 & 8  & 48 & 18 \\
DMU7  & 29 & 12 & 7  & 42 & 16 \\
DMU8  & 34 & 16 & 10 & 60 & 25 \\
DMU9  & 33 & 14 & 9  & 58 & 23 \\
DMU10 & 36 & 17 & 11 & 70 & 30 \\
\end{tabular}
\end{center}
Apply EWM to find the weights of the five inputs.

\subsection*{MATLAB Code}
\begin{lstlisting}
data = [
    30 12  8 50 20;
    35 15 10 65 28;
    28 11  7 40 15;
    32 14  9 55 22;
    40 18 11 80 35;
    31 13  8 48 18;
    29 12  7 42 16;
    34 16 10 60 25;
    33 14  9 58 23;
    36 17 11 70 30
];

X_norm = data ./ sum(data);

[m, n] = size(X_norm);
k = 1/log(m);
entropy = -k * sum(X_norm .* log(X_norm + eps));

d = 1 - entropy;
w = d / sum(d);

disp('Entropy weights for inputs [Faculty, Funding, Classrooms, Grad Students, Publications]:');
disp(w);
\end{lstlisting}

\subsection*{Output Screenshot}
\begin{figure}[h]
  \centering
  \includegraphics[width=0.7\textwidth]{ss3.png}
  \caption{MATLAB command window output for Q3}
\end{figure}

\end{document}